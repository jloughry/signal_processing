%
% Paper for the journal Signal Processing, written by Ireneusz Kubiak and
% edited by Joe Loughry. For information on this file please contact Joe
% Loughry at Tel. +1 720 277 7800 (time zone GMT minus 7 hours) or Email:
% Joe.Loughry@cs.du.edu or mailto:Joe.Loughry@gmail.com
%

\documentclass[10pt,a4paper]{article}

\usepackage[english,british]{babel}
\usepackage{graphicx}
\usepackage[obeyspaces,hyphens]{url}
\newcommand{\URL}[1]{$\langle$\url{#1}$\rangle$}
\usepackage[plainpages=false,pdfpagelabels]{hyperref}
\usepackage{setspace}
\usepackage{subfig}
% for typesetting the Romanian S-comma \cb{}
\usepackage{combelow}

\begin{document}

\title{LED Arrays of Laser Printers as Sources of Valuable Emissions for
Electromagnetic Penetration Process
}

\author{Ireneusz Kubiak \and Joe Loughry}

\maketitle

\doublespacing

\begin{abstract}
	Protection of information against electromagnetic eavesdropping is an
important issue. Information may be derivable from the shape of an unintended
electromagnetic signal. The resulting electromagnetic emanations can be
correlated with processing of classified information. The problem extends to
computer printers. This article presents a technical analysis of LED arrays
used in monochrome computer printers and their contribution to unintentional
electromagnetic emanations. We analysed two printers from different
manufacturers, designated $A$ and $B$. The forms of useful signals and their
dependence on parameters of printing data are presented. Analyses were based
on realistic type sizes and distribution of glyphs. Pictures were
reconstructed from received radio frequency (RF) emanations. We observed
differences in
legibility of information receivable at a distance that we attribute to
different ways used by printer designers to control the LED arrays,
particularly the difference between relatively high voltage single-ended
waveforms and lower-voltage differential signals. To decode the compromising
emanations required knowledge of---or guessing---printer operating parameters
including resolution, printing speed, and paper size. The optimal RF bandwidth
for detecting individual pixels has been determined. Measurements were
carried out across differences in construction and control of the LED arrays
in tested printers, and the levels of RF emissions compared for selected
operating modes (fast, high quality, or toner saving mode) of the printing
device.

\end{abstract}

\textbf{Keywords:} LED array, laser printer, unintentional emission,
compromising emanations, electromagnetic eavesdropping, electromagnetic
infiltration, recognition and reconstruction, non-invasive data acquisition.

\section{Introduction}

Printers are one of the basic elements of a computer system. They translate
the electronic form of processed data into graphical form during the printing
process. As with every electronic device, printers are sources of
electromagnetic emanations. Besides control signals, which carry no
information ({\it e.g.}, directing the operation of stepper motors or
heaters), there are other signals (useful signals) that are correlated with
the information being processed. Such emissions are called `sensitive' or
`valuable' or `compromising' emanations from the point of view of
electromagnetic protection of processed information. Processed data may
be information displayed on a computer screen or printed (Figure
\ref{figure:Figure_01}).

\begin{figure}[!t]
    \centering
    \includegraphics[width=2.5in]{graphics/Figure" "01.jpg}
    \caption{Laser printer as a source of valuable emissions.}
    \label{figure:Figure_01}
\end{figure}

Like other devices included in a computer system \cite{Kuhn2002,Kubiak2016a},
the printer can be subject to electromagnetic infiltration, or eavesdropping
\cite{Ketenci2015a,Kubiak2016b}. Therefore, efforts to reduce the level of
susceptibility to electromagnetic eavesdropping initiated for such devices.
Organisational and technical solutions \cite{Kubiak2006a} are the most
often-used methods for limiting infiltration
sensitivity of devices.\footnote{An example of an organisational solution
might be establishment of a `control zone' around susceptible devices,
relying on distance to attenuate signals below levels that can be received
outside the control zone.} Technical solutions are limited to changes in the
design of devices that typically increase the cost of such devices and
sometimes limit their functionality. Therefore, it is desirable to find
solutions that avoid these drawbacks and at the same time allow `safe'
processing of classified information \cite{Wasfy2011a,Goel2012a}.

One technical method that is commonly used in the field of electromagnetic
compatibility---both to reduce the amount of electromagnetic interference
emitted from the device and the susceptibility of the device to
electromagnetic disturbance---is the use of differential signals. Analysis
of useful signals and control signals \cite{Kubiak2017d} in the operation of
LED arrays
used in printers shows that such a design was used by the $B$ printer in the
operation of its photoconductor exposure system. Is this sufficient, however,
to foil non-invasive information gathering? Research and results are
presented in this article.

\begin{figure*}[!t]
    \centering
    \subfloat[Printer $A$]{\includegraphics[width=2in]{graphics/Figure" "02a.jpg}%
    \label{figure:Figure_02a}}
    \hfil
    \subfloat[Printer $B$]{\includegraphics[width=2in]{graphics/Figure" "02b.jpg}%
    \label{figure:Figure_02b}}
    \caption{Two printers, $A$ and $B$, were tested for sensitive emissions.}
    \label{figure:Figure_02}
\end{figure*}

The clear answer is that the solution adopted in the design of the $B$
printer (Figure \ref{figure:Figure_02b}) significantly reduces the
susceptibility of the device to infiltration, in comparison to the $A$
printer (Figure \ref{figure:Figure_02a}). Moreover, the level of
electromagnetic [emission?] of the $A$ printer is higher than that of
typical single and dual diode laser printers \cite{Kubiak2014b}.

% Watch out that the combined width of all the subfigures on a
% line do not exceed the text width or a line break will occur.

\singlespacing

\bibliographystyle{plain}
\bibliography{consolidated_bibtex_file}

\end{document}

