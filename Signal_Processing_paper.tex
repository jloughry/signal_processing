%
% This is a template for a new research paper. For information on this file
% please contact Joe Loughry at Tel. +1 720 277 7800 (time zone GMT minus 7
% hours) or Email: Joe.Loughry@cs.du.edu or mailto:Joe.Loughry@gmail.com
%

\documentclass[10pt,a4paper]{article}

\usepackage[english,british]{babel}
\usepackage{graphicx}
\usepackage[obeyspaces,hyphens]{url}
\newcommand{\URL}[1]{$\langle$\url{#1}$\rangle$}
\usepackage[plainpages=false,pdfpagelabels]{hyperref}
\usepackage{setspace}

\begin{document}

\title{LED arrays in laser printers are sources of compromising emanations for RF
electromagnetic attack
}

\author{Ireneusz Kubiak \and Joe Loughry}

\maketitle

\doublespacing

\begin{abstract}
	Protection of information against electromagnetic eavesdropping is an
important issue. Information may be derivable from the shape of an uninteinded
electromagnetic signal. The resulting electromagnetic emanations can be
correlated with processing of classified information. The problem extends to
computer printers. This article presents a technical analysis of LED arrays
used in monochrome computer printers and their contribution to unintentional
electromagnetic emanations. We analyzed two printers from different
manufacturers, designated $A$ and $B$. The forms of useful signals and their
dependence on parameters of printing data are presented. Analyses were based
on realistic type sizes and distribution of glyphs. Pictures were
reconstructed from received RF emanations. We observed differences in
legibility of information receivable at a distance that we attribute to
different ways used by printer designers to control the LED arrays,
particularly the differnce between relatively high voltage single-ended
waveforms and lower-voltage differential signals. To decode the compromising
emanations required knowledge of or guessing printer operating parameters
including resolution, printing speed, and paper size. The optimal RF bandwith
for detecting individual pixels has been determined. Measurements were
carried out across differences in construction and control of the LED arrays
in tested printers, and the levels of RM emissions compared for selected
operating modes (fast, high quality, or toner saving mode) of the printing
device.

\end{abstract}

\textbf{Keywords:} LED array, laser printer, unintentional emission,
compromising emanations, electromagnetic eavesdropping, electromagnetic
infiltration, recognition and reconstruction, non-invasive data acquisition.

\section{Introduction}

\begin{figure}[!t]
    \centering
    \includegraphics[width=2.5in]{graphics/Figure" "01.jpg}
    \caption{This is the caption.}
    \label{figure:Figure_01}
\end{figure}

% An example of a double column floating figure using two subfigures.
% (The subfig.sty package must be loaded for this to work.)
% The subfigure \label commands are set within each subfloat command,
% and the \label for the overall figure must come after \caption.
% \hfil is used as a separator to get equal spacing.
% Watch out that the combined width of all the subfigures on a
% line do not exceed the text width or a line break will occur.
%
%\begin{figure*}[!t]
%\centering
%\subfloat[Case I]{\includegraphics[width=2.5in]{box}%
%\label{fig_first_case}}
%\hfil
%\subfloat[Case II]{\includegraphics[width=2.5in]{box}%
%\label{fig_second_case}}
%\caption{Simulation results for the network.}
%\label{fig_sim}
%\end{figure*}

\singlespacing

\bibliographystyle{plain}
\bibliography{consolidated_bibtex_file}

\end{document}

